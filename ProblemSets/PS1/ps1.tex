% Created 2019-01-14 Mon 01:00
% Intended LaTeX compiler: pdflatex
\documentclass[12pt, letterpaper]{paper}
\usepackage[utf8]{inputenc}
\usepackage[T1]{fontenc}
\usepackage{graphicx}
\usepackage{grffile}
\usepackage{longtable}
\usepackage{wrapfig}
\usepackage{rotating}
\usepackage[normalem]{ulem}
\usepackage{amsmath}
\usepackage{textcomp}
\usepackage{amssymb}
\usepackage{capt-of}
\usepackage{hyperref}
\usepackage{natbib}
\usepackage{tikz}
\usepackage{bm}
\usepackage{minted}
\usepackage{natbib}
\usepackage[margin=1in]{geometry}
\author{Timothy Schwieg}
\date{\today}
\title{Structural Estimation Pset 1}
\hypersetup{
 pdfauthor={Timothy Schwieg},
 pdftitle={Structural Estimation Pset 1},
 pdfkeywords={},
 pdfsubject={},
 pdfcreator={Emacs 26.1 (Org mode 9.1.9)}, 
 pdflang={English}}
\begin{document}

\maketitle

\section{Structural Estimation is better than reduced-form}
\label{sec:orga04345e}
While Structural Estimation does rely upon assumptions, particularly
about the exact form of the endogeneity, it is explicit about what
these are, rather than implicitly making them. The form of these
assumptions is also usually economic rather than statistical. Economic
assumptions are easier to evaluate \emph{a priori} than statistical ones,
as these statistical assumptions must have theory applied to them to
be evaluated. 

The major problem with reduced-form estimation is that it has moved
its "extreme" assumptions from parametric assumptions such as function
forms and distributions to implicit instrument assumptions. As Keane
states, instruments can only be viewed as exogenous when viewed
through the lens of some theory, so they can never be taken as
"letting the data speak for itself." However these assumptions are now
implicit and not as spelled out, so it can be very easy to miss that
the instruments perhaps are not as powerful as they seem, as is the
case of draft lotteries still being endogenous to earnings. 

As Keane says on page 5, "If the work is to guide future policy, it is
important to understand what mechanism was at work." That is,
structural estimation serves to estimate primitives, of which there is
an economic, rather than statistical interpretation. This allows for
prediction to made about future policy changes, which are not
achievable by reduced-form estimation. Rust echos this when he says
"the statistical model typically cannot tell us anything about
[behavioral] response[s] to hypothetical policy changes that have not
yet occurred." Only structural techniques are invariant to the Lucas
Critique.

Structural Estimation builds Causality into the model, rather than
trying to infer it "from the data." It is explicit where the
assumptions are made to derive this causality, rather than implicitly
made, such as monotonicty, or homogeneity among the
population. While this does lead to mis-specification error, that
problem is not removed by changing to the often-linear reduced form
estimations used by many economists. Even strongly non-linear reduced
form models such as gradient boosting and random forests suffer from
mis-specification error. 

\section{However, it is not free from problems related to Instruments}
\label{sec:orgfb386d8}
Structural estimation, particularly in the empirical IO literature is
quite reliant on the use of Instrumental variable estimation. This is
due to both simultaneity problems as well as signaling aspects such
as unobserved quality being correlated with price. Keane especially
treats assaults upon instruments as assaults upon reduced-form
econometrics, but these techniques are used within structural models
as well. 

The empirical IO literature is especially known for its use of bad
instruments, ones that are very difficult to treat as exogenous. The
famous Haussman instruments in particular are difficult to believe
in. They require the different markets to be close enough that they
are using the same distribution networks, and thus will be correlated
on cost, but different enough that there can be no common shocks to
demand causing the price changes. I find this extremely hard to
believe, particularly in their initial application: beer
sales. Neither side seems able to address the fact that some of the
problems that they lay at the feet of reduced-form estimation come
from structural estimation being done poorly.
\end{document}